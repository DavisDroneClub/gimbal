% !TEX TS-program = pdflatex
% !TEX encoding = UTF-8 Unicode

% This is a simple template for a LaTeX document using the "article" class.
% See "book", "report", "letter" for other types of document.

\documentclass[11pt]{article} % use larger type; default would be 10pt

\usepackage[utf8]{inputenc} % set input encoding (not needed with XeLaTeX)
\usepackage{graphicx}
\graphicspath{ {figures/} }
\usepackage{array}

%%% Examples of Article customizations
% These packages are optional, depending whether you want the features they provide.
% See the LaTeX Companion or other references for full information.

%%% PAGE DIMENSIONS
\usepackage{geometry} % to change the page dimensions
\geometry{letterpaper} % or letterpaper (US) or a5paper or....
% \geometry{margin=2in} % for example, change the margins to 2 inches all round
% \geometry{landscape} % set up the page for landscape
%   read geometry.pdf for detailed page layout information

\usepackage{graphicx} % support the \includegraphics command and options

% \usepackage[parfill]{parskip} % Activate to begin paragraphs with an empty line rather than an indent

%%% PACKAGES
\usepackage{booktabs} % for much better looking tables
\usepackage{array} % for better arrays (eg matrices) in maths
\usepackage{paralist} % very flexible & customisable lists (eg. enumerate/itemize, etc.)
\usepackage{verbatim} % adds environment for commenting out blocks of text & for better verbatim
\usepackage{subfig} % make it possible to include more than one captioned figure/table in a single float
% These packages are all incorporated in the memoir class to one degree or another...

%%% HEADERS & FOOTERS
\usepackage{fancyhdr} % This should be set AFTER setting up the page geometry
\pagestyle{fancy} % options: empty , plain , fancy
\renewcommand{\headrulewidth}{0pt} % customise the layout...
\lhead{}\chead{}\rhead{}
\lfoot{}\cfoot{\thepage}\rfoot{}

%%% SECTION TITLE APPEARANCE
\usepackage{sectsty}
\allsectionsfont{\sffamily\mdseries\upshape} % (See the fntguide.pdf for font help)
% (This matches ConTeXt defaults)

%%% ToC (table of contents) APPEARANCE
\usepackage[nottoc,notlof,notlot]{tocbibind} % Put the bibliography in the ToC
\usepackage[titles,subfigure]{tocloft} % Alter the style of the Table of Contents
\renewcommand{\cftsecfont}{\rmfamily\mdseries\upshape}
\renewcommand{\cftsecpagefont}{\rmfamily\mdseries\upshape} % No bold!

\input{arduinoLanguage.tex}

\usepackage{hyperref}
\hypersetup{
    colorlinks=true,
    linkcolor=blue,
    filecolor=magenta,      
    urlcolor=cyan,
}

\setlength{\parindent}{0pt}

\setcounter{tocdepth}{2}
%%% END Article customizations

%%% The "real" document content comes below...

\title{Electronic Gimbal Documentation}
\author{Davis Drone Club}
\date{\today} % Activate to display a given date or no date (if empty),
         % otherwise the current date is printed 


\begin{document}
\pagenumbering{gobble}
\maketitle
\begin{center}
\includegraphics[width = 0.5\textwidth]{Pictures/cover.png}
\end{center}

\newpage
\tableofcontents
\pagenumbering{roman}
\newpage
\listoffigures
\listoftables

\newpage
\pagenumbering{arabic}
\section{Introduction}

\newpage
\section{Mechanical Design}

Your text goes here.

\subsection{Sheet Metal Parts}

\subsection{3D Printed Parts}

\newpage
\section{Electrical Systems}

More text.

\subsection{Actuator}

\subsection{Microcontroller}

\newpage
\section{Software}



\subsection{Arduino Code}
The gimbal code is designed to run on the 16MHz Arduino Nano. The program consists of library and variable instantiation, setup procedure, loop procedure, and IMU interfacing methods.

\subsubsection{Libraries}
The Wire library is pre-installed with the Arduino IDE and provides methods for using the I2C communication protocol to interface with the MPU6050. The Servo library is also pre-installed with the Arduino IDE and provides methods for driving servo motors. \\

The PID\_v1 library can be downloaded from \href{https://github.com/br3ttb/Arduino-PID-Library}{the PID Library Github}.\\

\lstinputlisting[language=Arduino, firstline=17, lastline=19]{code/code.ino}

\subsubsection{Variable Instantiation}
Variable instantiations begin with user changeable fields:
\lstinputlisting[language=Arduino, firstline=21, lastline=27]{code/code.ino}

\subsection{Python Code}
\end{document}




